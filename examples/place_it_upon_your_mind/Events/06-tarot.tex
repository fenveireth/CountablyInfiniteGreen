% Test unlocking a custom tarot card
% Demonstrate more advanced effect queueing
% Demonstrate death message select
% Demonstrate connecting an event to a map icon

\eventModify{heartbonevalley_map}
\option [FenJump]
% \hidden{MY_VARIABLE == 1} % If this test fails, icon is not placed on map
% \if{MY_VARIABLE == 1} % If this test fails, icon is greyed out and inactive
\set{TEMP_P4A_SELECTED = 0}
\transition{fade} \go{fen_test_tarot}

% If you're testing this, remember you can't unlock a card twice. Reset the
% variable to see the animation again

\event{fen_test_tarot}
\image{Promentory_Image}
\sound{base:Audio/P4_Music P4Mus}
\sound{base:Audio/P4_Ambience P4Amb}

If you jump, you'll get a Tarot card

\option Do a flip this time
\set{TEMP_P4A_SELECTED = 1} \go{fen_test_tarot2}


\event{fen_test_tarot2}

Successful takeoff

Gravity takes over
\set{TEMP_FALL = 1} \soundStop{P4Mus} \sound{base:Audio/P4_Fall falling}

% Using delays, you can queue several effect starts and stops on the same
% paragraph, even for the same channels
\option Use the ground to stop your fall
\set{DAMAGE += 30}
\set{TEMP_FALL = 2}
\soundStop{falling} \soundStop{P4Amb}
\effect{ScreenShakeInitializer shake} \sound{base:Audio/P4_Crash crash}
\effectStop{shake .5} \effectStop{crash 3}
\go{fen_test_tarot3}

\event{fen_test_tarot3}
Worth it

\option /EndEvent
% Because the card is unlocked with a \set instead of an \effect like the base
% game does, you can't set the unlock animation chained on a timer like they do
% The unlock animation blocks any other, and transitions, until done playing
\set{GLOBAL_TAROT_PETIT = 1}
% The base game uses numerics for DEATH. Please don't
% This connects to the values / conditions in Data/death_information.xml
\set{DEATH = devtest_falling}
\transition{whiteout} \go{death}
